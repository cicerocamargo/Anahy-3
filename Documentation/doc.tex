\documentclass[12pt]{article}
\usepackage{sbc-template}
\usepackage{graphicx,url}
\usepackage[brazil]{babel}   
\usepackage[latin1]{inputenc}  

\title{Documentação Oficial do Anahy 3}

\author{Gerson Geraldo H. Cavalheiro, Cícero Augusto de S. Camargo, Alan S. de Araújo}

\address{Programa de Pós-Graduação em Computação (PPGC)\\
	Universidade Federal de Pelotas (UFPEL)
  \email{\{cadscamargo,asdaraujo,gerson.cavalheiro\}@inf.ufpel.edu.br}
}

\begin{document}

\maketitle

\section{Introdução} % (fold)
\label{sec:introducao}

% section introducao (end)

\section{Modelo de execução} % (fold)
\label{sec:modelo_de_execucao}

% section sec:modelo_de_execucao (end)

\section{Código Fonte} % (fold)
\label{sec:codigo}

O código fonte de Anahy 3 é escrito em C++, usando uma modelagem orientada a objetos e threads POSIX para a execução de múltiplos threads no nível de sistema.

\subsection{Classes} % (fold)
\label{sub:classes}


\subsubsection{Job}

\paragraph{JobId}


\paragraph{JobAttributes}


\subsubsection{VirtualProcessor}


\subsubsection{Daemon}

\paragraph{SchedulingOperation} 


\subsubsection{AnahyVM} 

% subsection classes (end)

% section codigo_fonte (end)

\section{Biblioteca athread.h} % (fold)
\label{sec:athread}

A biblioteca athread.h fornece uma interface de programação similar a de pthread para  bla bla bla

\paragraph{aInit}

\paragraph{athread\_create}

\paragraph{athread\_join}

\paragraph{athread\_exit}

\paragraph{aTerminate}

% section biblioteca_athread (end)

\section{Conclusão} % (fold)
\label{sec:conclusao}

Anahy é bom demais!

% section conclusao (end)

\bibliographystyle{sbc}
\bibliography{bibs}

\end{document}

