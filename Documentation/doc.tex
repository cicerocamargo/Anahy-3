\documentclass[12pt]{article}
\usepackage{sbc-template}
\usepackage{graphicx,url}
\usepackage[brazil]{babel}   
\usepackage[utf8]{inputenc}  

\title{Documentação Oficial do Anahy 3}

\author{Gerson Geraldo H. Cavalheiro, Cícero Augusto de S. Camargo, Alan S. de Araújo}

\address{Programa de Pós-Graduação em Computação (PPGC)\\
	Universidade Federal de Pelotas (UFPEL)
  \email{\{cadscamargo,asdaraujo,gerson.cavalheiro\}@inf.ufpel.edu.br}
}

\begin{document}

\maketitle

\section{Introdução} % (fold)
\label{sec:introducao}

% section introducao (end)

\section{Modelo de execução} % (fold)
\label{sec:modelo_de_execucao}

% section sec:modelo_de_execucao (end)

\section{Código Fonte} % (fold)
\label{sec:codigo}

O código fonte de Anahy 3 é escrito em C++, usando uma modelagem orientada a objetos e threads POSIX para a execução de múltiplos threads no nível de sistema.

\subsection{Classes} % (fold)
\label{sub:classes}


\subsubsection*{Job}

\subsubsection*{JobId}


\subsubsection*{JobAttributes}


\subsubsection*{VirtualProcessor}


\subsubsection*{Daemon}


\subsubsection*{SchedulingOperation} 


\subsubsection*{AnahyVM} 

% subsection classes (end)

% section codigo_fonte (end)

\section{Biblioteca athread.h} % (fold)
\label{sec:athread}

A biblioteca \textbf{athread.h} oferece uma programação a alto nível para lançar atividades concorrentes, similar a interface do 
padrão POSIX Thread. A interface fornece um modelo \emph{fork/join} para descrever programas em termos de \emph{threads}. Uma 
camada intermediária entre \textbf{athread.h} e o núcleo do programa é responsável por identificar a concorrência fornecida 
a partir da infertace em pequenas partes, chamadas tarefas e implementadas pela classe \emph{Job}. A partir disso é possível
construir um DCG para representar as tarefas do programa. Esta camada intermediária é implementada na classe \emph{AnahyVM}.

\subsubsection*{aInit}

Este método coleta dados fornecidos pelo programador para configurar o ambiente de execução. A\_INIT tem a seguinte definição: 
\texttt{void aInit(int argc, char** argv)}, os dados passados por linha de comando são: número de processadores virtuais, modo de 
execução do ambiente, modelo de escalonamento das tarefas e mais alguns atributos que serão apresentados adiante.

\subsubsection*{athread\_create}

O A\_THREAD\_CREATE lança as atividaes concorrentes ao ambiente de execução. Este método fornece a seguinte definição: 
\texttt{int athread\_create(athread\_t* thid, athread\_attr\_t* attr, pfunc function, void\* args)}. Novos \emph{threads} são criados
para executarem a função em \texttt{pfunc function} que os descreve. Os dados de entrada para função são especificados em \texttt{args}.
O parâmetro \texttt{thid} é usado para identificar o novo \emph{thread} criado. O argumento \texttt{attr} especifica os atributos dos 
\emph{threads}, sendo estes, custos de computação, número de \emph{joins} sofridos entre outros que serão vistos na seção Atributos.

\subsubsection*{athread\_join}

A\_THREAD\_JOIN fornece a definição: \texttt{int athread\_join(athread\_t thid, void** result)}. Cada \emph{thread} identificada por
\texttt{thid} realiza uma operação de sincronização e \texttt{result} guarda a área de memória com o resultado final do \emph{thread}, 
esse dado é definido pela primitiva ATHREAD\_EXIT.

\subsubsection*{athread\_exit}

Este método cuja definição é \texttt{void athread\_exit(void* value\_ptr)} é chamado ao final de cada função que define um \emph{thread},
onde \texttt{value\_ptr} guarda o resultado final da execução em uma área de memória própria da tarefa.

\subsubsection*{aTerminate}

A\_TERMINATE é definida por \texttt{void aTerminate()} e é chamada no final da execução do programa para liberar a área de memória 
utilizada pelo ambiente. Ao término da execução do programa a função redefinir possíveis atributos do sistema que foram alterados 
durante a execução da aplicação.


% section biblioteca_athread (end)

\section{Conclusão} % (fold)
\label{sec:conclusao}

Anahy é bom demais!

% section conclusao (end)

\bibliographystyle{sbc}
\bibliography{bibs}

\end{document}

