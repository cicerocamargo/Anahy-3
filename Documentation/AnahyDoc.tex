\documentclass[12pt]{article} usepackage{sbc-template} usepackage{graphicx,url}
\usepackage[brazil]{babel}


\title{Documentação Oficial do Anahy 3}

\author{Gerson Geraldo H. Cavalheiro, Cícero Augusto de S. Camargo, Alan S. de
\Araújo}

\address{Programa de Pós-Graduação em Computação (PPGC)\\ Universidade Federal
de Pelotas(UFPEL)
\email{\{cadscamargo,asdaraujo,gerson.cavalheiro\}@inf.ufpel.edu.br} }

\begin{document}

\maketitle

\section{Introdução} % (fold) label{sec:introducao}

% section introducao (end)

\section{Modelo de execução} % (fold) label{sec:modelo_de_execucao}

Uma programa em Anahy pode ser descrito, resumidamente, por um grafo cíclico de
jobs (threads de usuário) e processadores virtuais que consomem os jobs deste
grafo em alguma ordem que respeite as dependências estabelecidas entre estes.

O fluxo normal de uma aplicação que usa o ambiente Anahy consiste da execução
da função \textt{main} em um único thread, até que seja o ambiente seja
explicitamente iniciado pela chamada de uma função da API para tal. Feito
isso, é possível criar e sincronizar jobs dinamicamente, de forma irrestrita
na maneira que as dependências entre estes são organizadas, desde que os jobs
criados diretamente a partir do thread main sejam sincronizados também no
thread main. Após o último job sincronizado a partir do thread main concluir
sua execução e antes do fim do thread main, o programador deve encerrar o
ambiente Anahy de maneira adequada.

% section sec:modelo_de_execucao (end)

\section{Código Fonte} % (fold) label{sec:codigo}

O código fonte de Anahy 3 é escrito em C++, usando uma modelagem orientada a
objetos e threads POSIX~\cite{NicholsPthreads} para a execução multithread no
nível de sistema.

\subsection{Classes} % (fold) label{sub:classes}

A evolução de um programa que use o ambiente multithread de Anahy se dá pela
criação dinâmica de instâncias das classes que serão descritas a seguir e a consequente troca de
mensagens entre estes objetos.

\subsubsection{Job}

Um job é a unidade básica de execução de Anahy. Programadores espicificam jobs indiretamente, através da interface de programação (API) que será descrita na Seção~\ref{sec:athread}. Um novo job é criado, ele recebe um ponteiro para uma função C/C++ que poderá ser executada em paralelo com o código subsequente à chamada da API que gerou este movo. A função pela qual um job deverá iniciar sua execução, e demais chamadas realizadas a partir dessa, podem conter chamadas da API para criar e/ou sincronizar jobs, gerando novas dependências no grafo de jobs. Assim, um job possui ponteiros para seu pai (nulo, se o job for criado a partir do thread main), seus filhos, o Processador Virtual que o criou, além de uma variáve que indica seu estado e outros objetos que serão descritos a seguir.

\paragraph{JobId}

Um job possui uma identificação única (um objeto, instância de JobId) no ambiente Anahy. Os Processadores Virtuais são as entidades responsáveis por criar diretamente um Job e, assim, seu respectivo JobId. Este JobId é composto pelo id do Processador Virtual que o criou e um número de série correspondente ao número de jobs criados naquele Processador Virtual até o momento.

\paragraph{JobAttributes}

Um job possui um objeto que descreve atributos, os quais dizem respeito às restrições que o programador deseja impor à execução do mesmo. (\textbf{?????})

\subsubsection{VirtualProcessor}

Processadores Virtuais (Virtual Processors) são objetos criado pela máquina virtual do ambiente Anahy em número estabelecido pelo programador através da API. Estes elementos, estão sempre associados a um Daemon, ao qual se referem quando estão ociosos e necessitam de um job para executar. Processadores Virtuais também recebem mensagens a partir das funções da API, para criar um novo job, ou suspender a execução do job em execução e executar outro.

Processadores Virtuais notificam seu Daemon associado sempre que criam um novo job (operação \textit{NewJob}), encerram a execução de um job (operação \textit{EndJob}) ou necessitam de outro job para executar (operação \textit{GetJob}). Nesta última situação, o Processador Virtual deve permanecer bloqueado até que sua solicitação seja atendida.

A máquina virtual de Anahy associa um objeto do tipo Virtual Processor ao thread main, ou seja, entre o início e término do ambiente Anahy, o thread main age como um processador virtual durante as chamadas às funções da API.

\subsubsection{Daemon}

O Daemon é um objeto cuja finalidade é atender as solicitações dos Processadores Associados a ele, as quais manipulam o grafo de jbos. Solicitações enviadas a um Daemon são atendidas por ordem de chegada e, caso uma destas seja do tipo \textit{GetJob} e não possa ser atendida, é colocadas em uma segunda fila de solicitações suspensas. Caso ocorra alguma operação do tipo \textit{NewJob}, o Daemon atualiza o grafo e, em seguida, atende uma das solicitações suspensas.

\paragraph{SchedulingOperation}


\subsubsection{AnahyVM}

% subsection classes (end)

% section codigo_fonte (end)

\section{Biblioteca athread.h} % (fold)
\label{sec:athread}

A biblioteca \textbf{athread.h} oferece uma programação a alto nível para lançar atividades concorrentes, similar a interface do 
padrão POSIX Thread. A interface fornece um modelo \emph{fork/join} para descrever programas em termos de \emph{threads}. Uma 
camada intermediária entre \textbf{athread.h} e o núcleo do programa é responsável por identificar a concorrência fornecida 
a partir da infertace em pequenas partes, chamadas tarefas e implementadas pela classe \emph{Job}. A partir disso é possível
construir um DCG para representar as tarefas do programa. Esta camada intermediária é implementada na classe \emph{AnahyVM}.

\subsubsection*{aInit}

Este método coleta dados fornecidos pelo programador para configurar o ambiente de execução. A\_INIT tem a seguinte definição: 
\texttt{void aInit(int argc, char** argv)}, os dados passados por linha de comando são: número de processadores virtuais, modo de 
execução do ambiente, modelo de escalonamento das tarefas e mais alguns atributos que serão apresentados adiante.

\subsubsection*{athread\_create}

O A\_THREAD\_CREATE lança as atividaes concorrentes ao ambiente de execução. Este método fornece a seguinte definição: 
\texttt{int athread\_create(athread\_t* thid, athread\_attr\_t* attr, pfunc function, void\* args)}. Novos \emph{threads} são criados
para executarem a função em \texttt{pfunc function} que os descreve. Os dados de entrada para função são especificados em \texttt{args}.
O parâmetro \texttt{thid} é usado para identificar o novo \emph{thread} criado. O argumento \texttt{attr} especifica os atributos dos 
\emph{threads}, sendo estes, custos de computação, número de \emph{joins} sofridos entre outros que serão vistos na seção Atributos.

\subsubsection*{athread\_join}

A\_THREAD\_JOIN fornece a definição: \texttt{int athread\_join(athread\_t thid, void** result)}. Cada \emph{thread} identificada por
\texttt{thid} realiza uma operação de sincronização e \texttt{result} guarda a área de memória com o resultado final do \emph{thread}, 
esse dado é definido pela primitiva ATHREAD\_EXIT.

\subsubsection*{athread\_exit}

Este método cuja definição é \texttt{void athread\_exit(void* value\_ptr)} é chamado ao final de cada função que define um \emph{thread},
onde \texttt{value\_ptr} guarda o resultado final da execução em uma área de memória própria da tarefa.

\subsubsection*{aTerminate}

A\_TERMINATE é definida por \texttt{void aTerminate()} e é chamada no final da execução do programa para liberar a área de memória 
utilizada pelo ambiente. Ao término da execução do programa a função redefinir possíveis atributos do sistema que foram alterados 
durante a execução da aplicação.


% section biblioteca_athread (end)

\section{Conclusão} % (fold)
\label{sec:conclusao}

Anahy é bom demais!

% section conclusao (end)

\bibliographystyle{sbc}
\bibliography{bibs}

\end{document}